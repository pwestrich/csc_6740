
\documentclass[12pt]{article}
\usepackage[top=1in, bottom=1in, left=1in, right=1in]{geometry}
\usepackage{setspace}

\title{Term Paper Proposal}
\author{Philip Westrich}
\date{February 24, 2017}

\begin{document}

\maketitle

\vspace{-0.3in}
\noindent
\rule{\linewidth}{0.4pt}
\doublespacing

Modern high-performance machines scale up to thousands of nodes and tens of thousands of cores. In order to properly 
diagnose performance or logic issues, there is a need for accurate and efficient trace files generated by the process. 

Unfortunately there are two major problems when collecting trace files. First, if assembling and writing the data for 
the trace file is too inefficient, it can disrupt the flow and performance of the original task at hand. Second, the 
sheer amount of data collected by the trace system can be very difficult or even impossible to store and analyze for 
meaningful information due to their impressive size.

Therefore, trace file generators for large parallel and/or distributed systems must be picky in the information they 
collect and store, only writing what is absolutely needed. They also must be efficient in doing so, as it is not 
desirable to add too much additional overhead to the main task at hand.

In my term paper for this course, I would like to write about the difficulties in generating those trace files as well 
as present and evaluate the solutions and techniques that others have introduced. This also should help me along with 
my thesis as I believe this will become a major component of it.

\end{document}
