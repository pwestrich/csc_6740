
\documentclass[12pt]{article}
\usepackage[margin=1in]{geometry}
\usepackage{setspace}

\title{Determining the Overhead of the EZTrace System}
\author{Philip M. Westrich}
\date{April 28, 2017}

\begin{document}

\maketitle
\vspace{-0.3in}\noindent\rule{\linewidth}{0.4pt}
\doublespacing

\begin{abstract}

    High performance computing plays an integral role in modern science. Many will use large scale simulations to test 
    things from new theories in astrophysics to determining the next week's weather. These simulations depend on the 
    programs written to carry them out; if they are incorrect or inefficient, they may possibly hinder their progress 
    rather than help it.
    
    The people who run these simulations always would like to make them faster; no one likes long wait times. Two methods 
    used to determine the inefficiencies in large parallel or distributed systems are tracing and logging. However, they 
    both do not come for free. On the extreme end, parallel debuggers can slow down a program up to 1000 times!
    
    In this paper, we will test the tracing system EZTrace and determine how much of a performance penalty it introduces 
    into a system.\\
 
\end{abstract}

\vspace{-0.3in}\noindent\rule{\linewidth}{0.4pt}

\section{Introduction}

Large scale simulations are used all over the place to model hard to test phenomenon in modern science. As these models 
become more complex and the simulations used to predict according to them become more and more complex, they require more 
computational power, time, or both to produce accurate enough results.

In order to prevent resources from being wasted, it is important that the algorithms used are as efficient as possible. 
The difference between $ O(n^{2.5}) $ and $ O(n^3) $ in computation or communication could mean hundreds of hours per 
simulation wasted.

\subsection{Tracing and profiling}

Two related methods used to discern the reasons behind poor performance are tracing and profiling. 

\subsection{The EZTrace system}

These complex systems require complicated tools to analyze them, especially once parallel and distributed computing are 
introduced. One solution presented was the EZTrace performance analysis framework \cite{Trahay2011}.

\section{Testing methods}

This is how I describe how I tested things.

\section{Results}

This is where I describe and discuss my results.

\section{Conclusion}

This is where I sum everything up.

\bibliographystyle{acm}
\bibliography{bibliography.bib}

\end{document}
